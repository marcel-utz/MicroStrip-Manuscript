%!TEX root = microstrip.tex

% Review on Micro-Striplines and Micro-Slot probes
% Marcel Utz and Ali Yilmaz
% February 2016

\section{Introduction}

Nuclear magnetic resonance refers to the interaction of a system of nuclear spins exposed to a static 
magnetic field with an oscillatory field, usually in the radio frequency range (anywhere from 1 kHz to 
several GHz) \cite{Abragam:1961vg}. In the overwhelming majority of cases, the changes in the spin state that 
result from this interaction are read out through a voltage induced by the spin precession into a 
surrounding conductor. Even though alternative readout approaches have been demonstrated, and have 
their advantages in certain cases, this type of inductive detection has proven to be both robust and 
easy to implement. Indeed, while the earliest demonstrations of nuclear magnetic resonance, pioneered 
by Rabi,\cite{Rabi:1938tq} relied on the deflection atomic beams in inhomogeneous fields according to 
the spin state, NMR did not take off as a widely used tool until the invention of the direct induction 
method, independently discovered by Bloch\cite{Bloch:1946hk} and by Purcell, Torrey, and Pound\cite{Purcell:1946ft} in 1946. 

Among other advantages, inductive detection allows the use of the same structure for both excitation and detection of 
the nuclear spin precession. Particularly in the context of the Fourier spectroscopy method,\cite{Ernst:1966cr} this has 
become very useful. It is relatively easy to expose a sample to an oscillatory magnetic field by 
surrounding it with a suitable conductor, through which an alternating current is sent at the appropriate 
frequency. The precessing spins induce a measurable voltage in the same conductor. Of course, there are 
technical problems to be solved that arise due to the high power that is needed in some cases for 
excitation, while the induced voltages are very small and require exquisitely sensitive receivers. It 
is not uncommon for excitation RF power to approach several kW, whereas the power available for spin 
detection is typically of the order of only a few pW.

The earliest inductive NMR systems have almost exclusively relied on solenoid coils as excitation/
detection systems. As the applications of NMR have diversified, and new technologies have become 
available, other geometries have been explored. In particular, the advent of superconducting magnets 
with cylindrical bores has led to the development of saddle coils and related resonator geometries. 
Magnetic resonance imaging, in particular for medical applications, brought the need to accommodate 
much larger samples, which was met by the development of birdcage resonators. 
Hence, most of the current NMR detectors follow a roughly cylindrical form factor. There are some 
applications, though, which require a more planar geometry. In particular, the study of thin films, 
membranes, and interfaces is complicated in cylindrical detector systems. Special, flattened solenoid 
probes have been developed for the study of membrane proteins under solid-state NMR conditions
\cite{Bechinger:1991in}. As the magnetic fields, and, correspondingly, the Larmor frequencies used in MRI scanners 
have increased, uniform penetration of the radio frequency magnetic fields into the tissue has become 
more and more difficult. Since biological systems contain ionic solutions, non-conservative electric 
fields arising in the detector system need to be shielded from getting in contact with the tissue. The 
dielectric losses incurred from interaction between these electric fields and the tissue degrade the 
sensitivity of the detection, but they also lead to excessive power deposition in the tissue during 
excitation. As a solution, surface coils for localised MR imaging of human subjects based on circular 
or quadratically laid out strip lines have been proposed. \cite{Zhang:2001js}
NMR spectroscopy is an extremely versatile tool. Nuclear spins turn out to be excellent spies. They are 
well insulated from the  noisy electronic degrees of freedom to allow for long spin coherence life 
times, which is the basis of sharp spectral lines. At the same time, the nuclear Larmor frequency turns 
out to depend in subtle ways on the electronic environment. In combination, these two features make it 
possible to both accurately measure the nuclear Larmor frequency, and to interpret it in terms of the 
molecular environment that surrounds the nucleus. 


Compared to many other spectroscopic techniques, NMR suffers from a major drawback: its relative 
insensitivity. While UV/VIS techniques, in particular fluorescence, can detect signals from single 
molecules with relative ease, NMR typically requires of the order of $10^{15}$ spins to resonate within a 
narrow bandwidth (1 Hz or so) in order for the signal to be measurable. To some extent, this can be 
alleviated by long measurement times. Since the signal/noise ratio only grows with the square root of 
the measurement time, sensitivity still limits the application of NMR in practice.
The design of NMR detectors that offer optimal sensitivity has therefore been a long-standing research 
topic in magnetic resonance. Sensitivity is determined by the signal/noise ratio that can be obtained 
within a specified amount of time from a defined number of spins. Inductive detectors based on 
resistive metals invariably produce thermal noise. Under optimal conditions, where all non-intrinsic 
sources of noise have been eliminated by shielding, the blackbody radiation of the resonator structure 
itself leads to a noise voltage spectral density which is essentially independent of frequency, and 
scales proportionally to the square root of the ohmic resistance of the detector. The relationship 
between the NMR spin precession and the induced voltage signal has been discussed by Hoult and Richards 
in terms of the correspondence principle.\cite{hoult1975cfd} The  induced signal strength from a single spin depends on 
the normalised magnetic field (generated by the detector per unit current) at the location of the spin. 
Hence, efficient detectors need to be designed such that the magnetic field they generate per unit 
current is maximal. This conflicts with the requirement of low resistance, which is important to keep 
the noise voltage small, and an optimal compromise must be found in practice between the two.
It has been well known for about two decades that the mass sensitivity (i.e., the signal/noise ratio 
per spin) of inductive detectors is roughly inversely proportional to the detector size. This can be 
rationalised by examining the normalised magnetic field and the radio frequency resistance of a 
particular detector geometry as a function of its overall dimensions. For example, the magnetic field 
generated by a single circular loop of diameter d made of a wire of thickness h is given approximately 
by $H/I=1/(\pi d)$. If the geometry is scaled by a factor $\alpha$, the $H/I$ value therefore scales as 
$\alpha^{-1}$. At typical NMR frequencies, the skin depth in Cu amounts only to a few $\mu$m. Therefore, 
as both the wire diameter and the loop diameter are scaled by the same factor $\alpha$, the 
resistance of the structure remains roughly constant, as long as the wire diameter remains larger than 
the skin depth. As a result the signal to noise ratio is expected to scale roughly as $1/\alpha$. A 
similar argument can be made for solenoid coils. In practice, the observed scaling is weaker. Still, 
NMR detectors based on micro coils (i.e., with dimensions of tens to hundreds of $\mu$m) have been shown 
to provide very high mass sensitivities. This has formed the basis of hyphenated techniques, where 
upstream chromatographic separation is combined with downstream detection by an NMR system equipped 
with a micro detector.\cite{webb2005nmr}
Microfluidics is a rapidly expanding field of science and technology. The underlying idea is borrowed 
from micro electronics: to integrate complex functionality in a mostly two-dimensional layout, making 
use of efficient lithographic fabrication technologies. This lab-on-a-chip (LoC) approach has proven 
especially fruitful in enabling total analysis systems, which integrate sample preparation, 
chromatographic separation, and detection on single chip platform.\cite{Manz:1990vc}
Since lithographic techniques allow the accurate reproduction of complex and very highly resolved 
features, microfluidic systems can be designed to mimic highly complex environments with great control 
and accuracy. This enables the culture of biological systems under artificial and highly controlled 
conditions, while closely mimicking the natural environment. This has become an invaluable tool for the 
study of  differentiated cells, their development, and the interplay between different cell types. 
{**REFS needed!!**}
NMR spectroscopy is uniquely suited to observe metabolic processes in live systems. It therefore has 
great potential as an observation tool in microfluidic culture assays. However, in spite of significant 
efforts, its use in the context of microfluidic devices is not yet widespread. There are a number of 
reasons for this. On the one hand, the planar geometry of microfluidic devices is not easily combined 
with common NMR detectors, which are typically designed for a cylindrical sample. Another limitation is 
the poor sensitivity of NMR, which is exacerbated by the small sample amounts typically available in 
microfluidic systems. As will become apparent in the following, micro stripline detectors are of 
particular interest in this context, since they inherently follow a planar geometry, and they can offer 
extremely high mass sensitivity.
The first NMR probes based on a strip line geometry have been proposed in the context of magnetic 
resonance imaging. Their use for NMR spectroscopy at the microscope was first proposed by Maguire et al 
\cite{Maguire:2009wc} in 2004. They integrated a stripline containing a small slot into a radio 
frequency resonator. The slot leads to current crowding, and consequently to a very large $H/I$ value 
locally. This geometry provides excellent mass sensitivity, but the achieved spectral resolution was 
still relatively poor. A few years later, van Bentum and Kentgens and coworkers proposed a stripline 
detector based on a symmetric geometry with ground planes on either side of the strip line. By tapering 
the transition in the width of the strip line broadening of the resonance lines due to magnetic 
susceptibility artefacts could be largely reduced, \cite{Bart:2009er} leading to excellent performance 
in terms of resolution. While these probes were operated in a flow mode, with fixed capillaries acting 
as sample holders, this geometry has recently been modified by Finch et al. to a transmission line 
probe based on two identical planar conductors, which can accommodate an exchangeable microfluidic 
chip.
Looking into the future, strip line based detector geometries offer significant potential for further 
advances in miniaturisation. Their fabrication using lithographic techniques is straightforward, in 
contrast to intrinsically three-dimensional geometries such as solenoids, and there is no reason why 
they could not be successfully applied to detectors an order of magnitude or more smaller than the ones 
that have been demonstrated so far. Another exciting possibility is the use of stripline detectors in 
travelling wave mode, rather than as resonators carrying standing waves. Travelling wave NMR, which 
been demonstrated in the context of magnetic resonance imaging and (macroscopic) NMR spectroscopy, 
could have signifiant advantages at the micro scale, since it allows the spatial separation of the 
sample and the detection circuitry. 
The remainder of this chapter is organised as follows: first, some theoretical aspects of strip lines 
and strip-line based resonators are examined in section 2. Section 3 provides a chronological overview 
of the development of micro-NMR strip line and micro strip based detectors, and finally, section 4 
discusses some of the recently demonstrated applications.