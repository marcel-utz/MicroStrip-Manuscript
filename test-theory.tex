\subsection{~Striplines and Microslots: Basics and
Theory}\label{striplines-and-microslots-basics-and-theory}

\subsubsection{Definition of a
stripline}\label{definition-of-a-stripline}

The presence of conducting bodies imposes boundary conditions on the
free propagation of electromagnetic waves. For an ideal conductor (an
idealisation that certain metals, including Cu and Ag, approach quite
closely), the electric field vector must stay perpendicular to the
surface, while the magnetic field vector is required to stay parallel.
Wave guides are long metallic structures of (usually) constant cross
section. Electromagnetic waves of different types (modes) can propagate
in the longitudinal direction in such structures. They are classified as
transverse electric (TE) or transverse magnetic (TM) waves. In TE waves,
the electric field has no component in the longitudinal direction, while
the longitudinal magnetic field vanishes for TM waves. Both TE and TM
wave modes can only propagate if the wave length is smaller than the
lateral dimensions of the wave guide. This leads to a minimum frequency
(often referred to as the cutoff frequency) for propagating modes. Also,
the relationship between the wavelength and the frequency for TE and TM
modes is non-linear, leading to dispersion. An additional type of
propagating mode becomes possible if the walls of the wave guide are
divided into several (at least 2) mutually insulated sections. In this
case, an oscillatory voltage can be sustained between the separate
conductors. Under these conditions, a propagating modes exist
textem\{both\} the electric and magnetic fields are transverse. Such TEM
modes do not exhibit a cutoff frequency, and in general exhibit a linear
relationship between wavelength and frequency. Frequencies of interest
in magnetic resonance lie below 1.5 GHz. With usual dielectrics, this
yields wave lengths of 20 cm or more. Therefore, TEM modes are commonly
used in order to transport NMR signals, often in coaxial cables. This is
different in electron paramagnetic resonance (EPR), where frequencies up
to several hundred GHz occur. This requires the use of rectangular wave
guides, in some cases with corrugated interior surfaces. Fig XXX shows
some examples of common wave guide cross sections. Among these, the most
familiar to NMR spectroscopists is that of the coaxial cable. Planar
wave guide structures such as the microstrip (Fig XXXX b) and the
stripline (Fig XXX c) are conveniently implemented on printed circuit
boards, @Barret:1955ie and are very commonly used in the design of radio
frequency and microwave circuits. \#\#\# Characteristic Impedance and
Transport Characteristics For TEM modes, it is possible to attribute a
current and a voltage amplitude to the travelling wave by integrating
along the electric / magnetic field lines in the cross section. While
the absolute voltage and current amplitudes depend on the level of wave
excitation, their ratio (measured in Ohm) is a constant given purely by
the cross section geometry, and the dielectric and magnetic properties
of the insulating medium. This ratio is known as the characteristic
impedance \(Z\_0\) of the wave guide. Coaxial cables are commonly
designed for a characteristic impedance of 50 Ohm. \#\#\# Theory of TEM
Wave Modes

\begin{itemize}
\item
  Characteristic impedance of a single stripline
\item
  Steps in impedance
\item
  Insertion of a microslot \#\#\# Transmission Line Resonators
\item
  Dimensions and eigenfrequencies
\item
  Q factor and sensitivity
\item
\end{itemize}
