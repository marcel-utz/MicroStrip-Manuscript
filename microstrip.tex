%!TEX root = microstrip.tex

% Review on Micro-Striplines and Micro-Slot probes
% Marcel Utz and Ali Yilmaz
% February 2016


\documentclass[prb,twocolumn]{revtex4}

\begin{document}

\title{Micro-Striplines and Micro-Slot NMR Probes}
\author{Ali Yilmaz and Marcel Utz}
\affiliation{School of Chemistry, University of Southampton, United Kingdom}
\date{DRAFT: \today}

\begin{abstract}
	Striplines are commonly used to transport signals in microwave and radio frequency circuits. They 
	are easily implemented on printed circuit boards. Over the past two decades, wave guides 
	with planar geometry have found increasing use as inductive detectors in miniaturised nuclear
	magnetic resonance systems. In this article, we give a brief overview of the theory of 
	striplines and resonators built from them, and we review the literature on their use in NMR
	spectroscopy.
\end{abstract}

	\maketitle
	
	\input introduction
	
	\input theory
	
	\input stripline-nmr
	
	\bibliographystyle{IEEEtran}
	\bibliography{literature}
	
\end{document}